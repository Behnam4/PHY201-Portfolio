\documentclass{article}
\oddsidemargin 0.0in
\textwidth 6.0in
\thispagestyle{empty}
\usepackage{import}
\usepackage{amsmath}


\begin{document}


{\bf Exercises set on matrices }\\

Each of the following exercises is worth 5 points (clarification: exercises are divided in groups, for convenience. Each item of the group -- which is often simply a single calculation --  is worth 5 points). 

\emph{Important reminder: to get full credit, all calculations must be done by hand, without using electronic devices. Steps must be reasonably detailed.  The use of electronic devices is allowed, but must be explicitly declared, and will result in partial credit. See syllabus for details. Also see syllabus for rules on collaborating with others on an assignment.   }  




\begin{enumerate}

\item  Determine which of the following matrices have inverses by computing their determinants (5 point per determinant). 

\begin{eqnarray*}
&&\rm{(a) }\left( 
\begin{array}{rrr}
3 & 1 & 5 \\ 
-1 & -3 & -1 \\ 
2 & 2 & 3
\end{array}
\right) ;\;\rm{(b) }\left( 
\begin{array}{rrr}
6 & -2 & 3 \\ 
1 & 1 & 1 \\ 
2 & -3 & 1
\end{array}
\right) ;\;\rm{(c) }\left( 
\begin{array}{rrr}
4 & 2 & 2 \\ 
-1 & 3 & -1 \\ 
3 & 4 & 5
\end{array}
\right) ;\; \\
&&\rm{(d)\ }\left( 
\begin{array}{rrrr}
2 & 3 & -1 & 1 \\ 
3 & -4 & 3 & -1 \\ 
2 & -1 & 1 & -3 \\ 
3 & 1 & -2 & 4
\end{array}
\right) ;\;

\end{eqnarray*}

\item Continue the previous exercise: for the matrices that have non-vanishing determinants, compute the inverses using a method of your choice (5 points for each inverse calculation). 


\item  Consider the following vectors (column matrices):
\[
\mathsf{A}=\left( 
\begin{array}{l}
2 \\ 
3
\end{array}
\right) ,\;\mathsf{B}=\left( 
\begin{array}{l}
1 \\ 
5
\end{array}
\right) ,
\]

Rotate ({\it i.e.}, find the necessary rotation matrix) $\mathsf{A}$ into the direction of $\mathsf{B}$.




\item  For each of the following matrices, (a) compute the trace, (b) write down the Hermitian adjoint, and (c) determine whether it is Hermitian, unitary or neither. \emph{( answering all the questions for each matrix is worth 5 points. Be careful and try to answer the questions in the most efficient way, avoiding unnecessary calculations)}. 
\[
\begin{array}{ccccc}
1. & A=\left( 
\begin{array}{ccc}
1 & 0 & -i \\ 
0 & -2 & 4-i \\ 
i & 4+i & 3
\end{array}
\right)  & \, & 2. & B=\left( 
\begin{array}{ccc}
\frac{1}{\sqrt{2}} & \frac{1}{\sqrt{6}} & \frac{1}{\sqrt{3}} \\ 
0 & -\frac{2}{\sqrt{6}} & \frac{1}{\sqrt{3}} \\ 
\frac{1}{\sqrt{2}} & -\frac{1}{\sqrt{6}} & -\frac{1}{\sqrt{3}}
\end{array}
\right)  \\ 
3. & C=\left( 
\begin{array}{lll}
2 & i & 0 \\ 
i & 1 & -i \\ 
0 & -i & 2
\end{array}
\right)  & \, & 4. & D=\left( 
\begin{array}{ccc}
\frac{i}{\sqrt{2}} & 0 & \frac{1}{\sqrt{2}} \\ 
0 & 1 & 0 \\ 
\frac{1}{\sqrt{2}} & 0 & \frac{i}{\sqrt{2}}
\end{array}
\right)  \\ 
5. & E=\left( 
\begin{array}{ccc}
\frac{i}{\sqrt{2}} & 0 & \frac{1}{\sqrt{2}} \\ 
0 & 1 & 0 \\ 
\frac{1}{\sqrt{2}} & 0 & -\frac{i}{\sqrt{2}}
\end{array}
\right)  & \, & 6. & F=\left( 
\begin{array}{ccc}
\frac{1}{\sqrt{2}} & 0 & \frac{i}{\sqrt{2}} \\ 
0 & 1 & 0 \\ 
-\frac{i}{\sqrt{2}} & 0 & \frac{1}{\sqrt{2}}
\end{array}
\right) 
\end{array}
\]


\item calculate the determinant of each of the matrices in the previous problem (each determinant is worth 5 points). 


\item Find the eigenvectors and eigenvalues of the
following matrices.  Eigenvectors do not need to be normalized. (5 points for each matrix)

\begin{enumerate}
\item  $\left( 
\begin{array}{rrr}
0 & 1 & 1 \\ 
1 & 0 & 1 \\ 
1 & 1 & 0
\end{array}
\right) $

\item  $\left( 
\begin{array}{rrr}
1 & 2 & 0 \\ 
1 & 0 & 1 \\ 
0 & 2 & 1
\end{array}
\right) $

\item  $\left( 
\begin{array}{rrr}
0 & 1 & i \\ 
1 & 0 & 1 \\ 
i & 1 & 0
\end{array}
\right) $

\item  $\left( 
\begin{array}{rrr}
0 & 1 & i \\ 
1 & 0 & 1 \\ 
-i & 1 & 0
\end{array}
\right) $
\end{enumerate}


\item Find the eigenvalues and \emph{orthonormal} sets of eigenvectors of the following matrices (5 points per matrix):

\begin{enumerate}
\item  $\left( 
\begin{array}{rrr}
2 & 0 & 0 \\ 
0 & 1 & 1 \\ 
0 & 1 & 1
\end{array}
\right) $

\item  $\left( 
\begin{array}{rrr}
1 & 1 & 1 \\ 
1 & 1 & 1 \\ 
1 & 1 & 1
\end{array}
\right) $

\item  $\left( 
\begin{array}{rrr}
5 & 0 & \sqrt{3} \\ 
0 & 3 & 0 \\ 
\sqrt{3} & 0 & 3
\end{array}
\right) $
\end{enumerate}




\end{enumerate}


\end{document}
