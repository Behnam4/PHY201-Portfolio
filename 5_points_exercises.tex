\documentclass{article}
\oddsidemargin 0.0in
\textwidth 6.0in
\thispagestyle{empty}
\usepackage{import}
\usepackage{amsmath}


\begin{document}


{\bf Exercises set on Vectors }\\

Each of the following exercises is worth 5 points (clarification: exercises are divided in groups, for convenience. Each item of the group -- which is often simply a single calculation --  is worth 5 points). 

\emph{Important reminder: to get full credit, all calculations must be done by hand, without using electronic devices. Steps must be reasonably detailed.  The use of electronic devices is allowed, but must be explicitly declared, and will result in partial credit. See syllabus for details. Also see syllabus for rules on collaborating with others on an assignment.   }  




\begin{enumerate}

\item You are given the following vectors in terms of their components in
the $\left\{ \mathbf{\hat{x}},\,\mathbf{\hat{y}},\mathbf{\, \hat{z}}\right\} $ basis: 
\[
\mathbf{A}=\mathbf{\hat{x}}+2\mathbf{\hat{y}-}2\mathbf{\hat{z}, \quad \quad B}=3\mathbf{\hat{x}}+\mathbf{\hat{y}+}2\mathbf{\hat{z}, \quad \quad C}=4\mathbf{\hat{x}}-\mathbf{\hat{y}+\hat{z}.}
\]
Take $\theta_{\rm AC}$ to be the angle between vectors $\mathbf{A}$ and $\mathbf{C}$, etc. 

\begin{enumerate}
\item Prove that these vectors are linearly independent. 

%\item compute $A,\;B$ and $C.$ (\emph{i.e.}, $A=\left| \mathbf{A}\right| $, etc.)

%\item compute $2\mathbf{A}+\mathbf{B},\;3\mathbf{C}-\mathbf{B,\;A}+\mathbf{B+C}$

%\item compute $\mathbf{A\cdot B,\;A\cdot C}$ and $\mathbf{B\cdot C}$

\item compute  $\mathbf{A\times B,\;A\times C}$ and $\mathbf{B\times C}$

\item compute (by hand) $\cos \theta_{\rm AB},\,\cos \theta_{\rm AC},\,\cos \theta_{\rm BC}$

\item  compute (by hand) $\sin \theta_{\rm AB},\,\sin \theta_{\rm AC},\,\sin \theta_{\rm BC}$

\item compute $\mathbf{A\cdot }\left( \mathbf{B\times C}\right) ,\;\mathbf{B\cdot }\left( \mathbf{A\times C}\right) $ and $\mathbf{C\cdot }\left( \mathbf{A\times B}\right)$

\item compute $\left( \mathbf{A\times B}\right) \times \mathbf{C+}\left( \mathbf{B\times C}\right) \times \mathbf{A+}\left( \mathbf{C\times A}\right) \times \mathbf{B}$
\end{enumerate}


\item Find an equation of the plane through the points $\left(
0,\,0,\,3\right) ,\;\left( 1,\,1,\,1\right) $ and $\left( -1,\,1,\,2\right) $.

\item Find the equation of the plane that contains the point $\left(
2,\,1,\,-4\right) $ and is perpendicular to the vector $3\mathbf{\hat{x}}-2\mathbf{\hat{y}}+\mathbf{\hat{z}}.$

\item Determine $b$ such that the line through $\left( 5,\,0,\,3\right) $ and $\left( -1,\,-10,\,b\right) $ will be perpendicular to $\left( \mathbf{\hat{x}+\hat{y}}\right) \mathbf{\times }\left( \mathbf{\hat{x}+\hat{z}}\right)$.

\item Use vectors to find the equation of the straight line containing the points $\left( 1,\,-2,\,4\right) $ and $\left( 6,\,1,\,1\right) .$ Write this equation in \emph{parametric form}, {\it i.e.}, in the form $\mathbf{r}= \mathbf{r}_{0}+t\mathbf{V}$, where $\mathbf{r}_{0}$ is the vector describing a given point on the line, $\mathbf{V}$ is a vector oriented along the line, and $t$ is a variable that has range $\left( -\infty ,\,\infty \right)$. 

\item 
Find the vector(s) having modulus equal to $\sqrt{13}$, which are orthogonal to the $y$-axis and parallel to the plane $3x - 2z -5=0$.

\item 
Considerate the lines $r$ and $s$, given by the  equations: 
\begin{equation}
r: 
\begin{cases} x-2= 0 \\ z+y=0    \end{cases} ; ~~~~~~   s: \begin{cases} x = t - 2  \\ y=t \\ 2t- 1   \end{cases}   
\end{equation}
%\end{equation}
Establish if it is possible to find a point $R$ located on $r$, and a point $S$ located on $s$, so that the line passing by $R$ and $S$ 
is parallel to the planes 
$\alpha : x - 3z + 2 = 0$ and  $\beta : 2x - y = 1$.

\item 
%Fissato un sistema di riferimento cartesiano dello spazio euclideo O, i, j, k, determinare equazioni parametriche della retta passante per P = (1, 3, 0), parallela al piano yz e al piano $x + y + z = 0$.

Find a parametric equation describing the line passing by the point $P = (1, 3, 0)$, and parallel to both the $x=0$ plane (i.e., the $yz$ plane) and the plane $x + y + z = 0$. 

\end{enumerate}


\end{document}
