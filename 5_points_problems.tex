\documentclass{article}
\oddsidemargin 0.0in
\textwidth 6.0in
\thispagestyle{empty}
\usepackage{import}


\begin{document}


{\bf Exercises set on Complex Numbers }\\

Each of the following exercises is worth 5 points (clarification: exercises are divided in groups, for convenience. Each item of the group -- which is often simply a single calculation --  is worth 5 points). 

\emph{Important reminder: to get full credit, all calculations must be done by hand, without using electronic devices. Steps must be reasonably detailed.  The use of electronic devices is allowed, but must be explicitly declared, and will result in partial credit. See syllabus for details. Also see syllabus for rules on collaborating with others on an assignment.   }  


\begin{enumerate}


\item  Perform  the following computations, reducing the result to
standard $a+ib$ form for complex numbers: 
\[
\begin{array}{lll}
13.\,\frac{\textstyle 1+i}{\textstyle 1-i} & 14.\,\frac{\textstyle 2+i}{\textstyle i-1} & 15.\,\frac{\textstyle 4+5i}{\textstyle 5-3i} \\ 
16.\,\frac{\textstyle 2+i}{\textstyle 3+2i}+\left( \frac{\textstyle i+4}{\textstyle 5i}\right)^{\ast} & 17.\,\frac{\textstyle i}{\textstyle 1+i}-\frac{\textstyle 1+i}{\textstyle i} & 18.\,\frac{\textstyle 3+5i}{\textstyle 5+3i}+4
\end{array}
\]


\item Perform the following calculations using the Euler (polar) form for \underline{each of the complex numbers} involved (in other words: take each number and transform it into polar form. Then, execute the operation given.).
\[
\begin{array}{lll}
1.\;\left( 1+i\right) \left( 1-i\right)  & 2.\;\frac{\textstyle 1+i}{\textstyle 1+i\sqrt{3}} & 3.\;i\left( 3+4i\right)  \\ 
4.\;\frac{\textstyle 1-3i}{\textstyle i} & 5.\;\frac{\textstyle 1+i}{\textstyle \sqrt{2}}\cdot \left( 1+i\sqrt{3}\right)  & 6.\;\left( 1+i\right) ^{5}
\end{array}
\]

\item Express the following quantities in Cartesian $\left(x+iy\right) $ form and Euler $\left( re^{i\theta}\right)$ form. Indicate where there are multiple values.

\begin{enumerate}
\item $\exp \left( 1+i\sqrt{3}\right)$

\item $\cos \left( 1+i\right)$

\item $\sin \left( 1+i\right)$

\item $\ln (2i)$

\item $\ln \left( 1-i\pi \right)$

\item $\cosh \left( 1+i\pi \right)$
\end{enumerate}

\item Compute the following roots and rational powers. Be sure to find all the solutions and plot them on an Argand plane. 
\[
\begin{array}{lll}
1.\;i^{1/2} & 2.\;\left( -i\right) ^{1/2} & 3.\;2^{1/3} \\ 
4.\;2^{2/3} & 5.\;(1+i\sqrt{3})^{3/5} & 6.\;\left( 1-i\right) ^{-2/3}
\end{array}
\]
\end{enumerate}


\end{document}
