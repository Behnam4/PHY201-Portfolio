\documentclass[fleqn]{article}
\oddsidemargin 0.0in
\textwidth 6.0in
\thispagestyle{empty}
\usepackage{import}
\usepackage{amsmath}
\usepackage{graphicx}
\usepackage[english]{babel}
\usepackage[utf8x]{inputenc}
\usepackage{float}
\usepackage[colorinlistoftodos]{todonotes}
\usepackage{setspace}

\doublespacing
\begin{document}

\begin{titlepage}

\newcommand{\HRule}{\rule{\linewidth}{0.5mm}} % Defines a new command for the horizontal lines, change thickness here

\center % Center everything on the page
 

\textsc{\LARGE Arizona State University}\\[1.5cm] % Name of your university/college

\textsc{\Large Mathematical Methods For Physics I }\\[0.5cm] % Major heading such as course name


\begin{figure}
  \includegraphics[width=\linewidth]{asu.png}
\end{figure}


\HRule \\[0.4cm]
{ \huge \bfseries Portfolio}\\[0.4cm] 
\HRule \\[1.5cm]
 
\textbf{Behnam Amiri}

\bigbreak

\textbf{Prof: Cecilia Lunardini (Grader. Kenna McRae)}

\bigbreak


\textbf{{\large \today}\\[2cm]}

\vfill % Fill the rest of the page with whitespace

\end{titlepage}

\huge \textbf{Academic integrity statement:}

\bigbreak

\Large I am aware of the course rules detailed in the syllabus and related course documents. I am also aware of Arizona State University’s policies and practices against plagiarism and other forms of academic dishonesty. I affirm that I have not given or received any unauthorized help on this assignment, and I have not used any unauthorized resources. All authorized help (from the instructor, or learning assistant), authorized collaborations (with classmates), and authorized resources are explicitly stated and described in detail in the present document.

\bigbreak

\Large This work is entirely my own, except when collaboration with classmates is explicitly declared, and I take full responsibility for it.

\bigbreak

\Large Signature and date:

\bigbreak

\bigbreak


\Large \textbf{ Behnam Amiri  September 22, 2020 }


\pagebreak

\textbf{Exercises set on matrices}

\begin{enumerate}

  \item Determine which of the following matrices have inverses by computing their determinants (5 point per determinant).

  \begin{eqnarray*}
    &&\rm{(a) }\left( 
    \begin{array}{rrr}
    3 & 1 & 5 \\ 
    -1 & -3 & -1 \\ 
    2 & 2 & 3
    \end{array}
    \right) ;\;\rm{(b) }\left( 
    \begin{array}{rrr}
    6 & -2 & 3 \\ 
    1 & 1 & 1 \\ 
    2 & -3 & 1
    \end{array}
    \right) ;\;\rm{(c) }\left( 
    \begin{array}{rrr}
    4 & 2 & 2 \\ 
    -1 & 3 & -1 \\ 
    3 & 4 & 5
    \end{array}
    \right) ;\; \\
    &&\rm{(d)\ }\left( 
    \begin{array}{rrrr}
    2 & 3 & -1 & 1 \\ 
    3 & -4 & 3 & -1 \\ 
    2 & -1 & 1 & -3 \\ 
    3 & 1 & -2 & 4
    \end{array}
    \right) ;\;
    \end{eqnarray*}

  \item Continue the previous exercise: for the matrices that have non-vanishing determinants, compute the inverses using a method of your choice (5 points for each inverse calculation). 
  
  \item  Consider the following vectors (column matrices):
    \[
    \mathsf{A}=\left( 
    \begin{array}{l}
    2 \\ 
    3
    \end{array}
    \right) ,\;\mathsf{B}=\left( 
    \begin{array}{l}
    1 \\ 
    5
    \end{array}
    \right) ,
    \]

  Rotate ({\it i.e.}, find the necessary rotation matrix) $\mathsf{A}$ into the direction of $\mathsf{B}$.


  \item  For each of the following matrices, (a) compute the trace, (b) write down the Hermitian adjoint, and (c) determine whether it is Hermitian, unitary or neither. \emph{( answering all the questions for each matrix is worth 5 points. Be careful and try to answer the questions in the most efficient way, avoiding unnecessary calculations)}. 
    \[
    \begin{array}{ccccc}
    1. & A=\left( 
    \begin{array}{ccc}
    1 & 0 & -i \\ 
    0 & -2 & 4-i \\ 
    i & 4+i & 3
    \end{array}
    \right)  & \, & 2. & B=\left( 
    \begin{array}{ccc}
    \frac{1}{\sqrt{2}} & \frac{1}{\sqrt{6}} & \frac{1}{\sqrt{3}} \\ 
    0 & -\frac{2}{\sqrt{6}} & \frac{1}{\sqrt{3}} \\ 
    \frac{1}{\sqrt{2}} & -\frac{1}{\sqrt{6}} & -\frac{1}{\sqrt{3}}
    \end{array}
    \right)  \\ 
    3. & C=\left( 
    \begin{array}{lll}
    2 & i & 0 \\ 
    i & 1 & -i \\ 
    0 & -i & 2
    \end{array}
    \right)  & \, & 4. & D=\left( 
    \begin{array}{ccc}
    \frac{i}{\sqrt{2}} & 0 & \frac{1}{\sqrt{2}} \\ 
    0 & 1 & 0 \\ 
    \frac{1}{\sqrt{2}} & 0 & \frac{i}{\sqrt{2}}
    \end{array}
    \right)  \\ 
    5. & E=\left( 
    \begin{array}{ccc}
    \frac{i}{\sqrt{2}} & 0 & \frac{1}{\sqrt{2}} \\ 
    0 & 1 & 0 \\ 
    \frac{1}{\sqrt{2}} & 0 & -\frac{i}{\sqrt{2}}
    \end{array}
    \right)  & \, & 6. & F=\left( 
    \begin{array}{ccc}
    \frac{1}{\sqrt{2}} & 0 & \frac{i}{\sqrt{2}} \\ 
    0 & 1 & 0 \\ 
    -\frac{i}{\sqrt{2}} & 0 & \frac{1}{\sqrt{2}}
    \end{array}
    \right) 
    \end{array}
    \]

  \item calculate the determinant of each of the matrices in the previous problem (each determinant is worth 5 points). 


  \item Find the eigenvectors and eigenvalues of the
    following matrices.  Eigenvectors do not need to be normalized. (5 points for each matrix)
    
  \begin{enumerate}
  \item  $\left( 
    \begin{array}{rrr}
    0 & 1 & 1 \\ 
    1 & 0 & 1 \\ 
    1 & 1 & 0
    \end{array}
    \right) $
    
    \item  $\left( 
    \begin{array}{rrr}
    1 & 2 & 0 \\ 
    1 & 0 & 1 \\ 
    0 & 2 & 1
    \end{array}
    \right) $
    
    \item  $\left( 
    \begin{array}{rrr}
    0 & 1 & i \\ 
    1 & 0 & 1 \\ 
    i & 1 & 0
    \end{array}
    \right) $
    
    \item  $\left( 
    \begin{array}{rrr}
    0 & 1 & i \\ 
    1 & 0 & 1 \\ 
    -i & 1 & 0
    \end{array}
    \right) $
    \end{enumerate}
    
    
  \item Find the eigenvalues and \emph{orthonormal} sets of eigenvectors of the following matrices (5 points per matrix):
    
  \begin{enumerate}
    \item  $\left( 
    \begin{array}{rrr}
    2 & 0 & 0 \\ 
    0 & 1 & 1 \\ 
    0 & 1 & 1
    \end{array}
    \right) $
    
    \item  $\left( 
    \begin{array}{rrr}
    1 & 1 & 1 \\ 
    1 & 1 & 1 \\ 
    1 & 1 & 1
    \end{array}
    \right) $
    
    \item  $\left( 
    \begin{array}{rrr}
    5 & 0 & \sqrt{3} \\ 
    0 & 3 & 0 \\ 
    \sqrt{3} & 0 & 3
    \end{array}
    \right) $
    \end{enumerate}
    

\end{enumerate}

% End of questions

\end{document}
